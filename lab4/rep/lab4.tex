\documentclass[a4paper,2pt]{report}

\usepackage[a4paper, total={6in, 9in}]{geometry}
\usepackage[table,xcdraw]{xcolor}
\usepackage{pdfpages}
\usepackage{indentfirst}
\usepackage{mathtools}
\usepackage{graphicx}
\usepackage{float}
\usepackage{subfig}
\usepackage{multirow}

\graphicspath{ {./img/} }


\setlength{\parskip}{6pt}

\begin{document}

\begin{titlepage}
    \begin{center}
        \vspace*{3cm}
 
        \LARGE
        \textbf{Instituto Superior Técnico}
        \vskip 0.4cm
 
        \Large{MEEC}
        \vskip 0.2cm

        \Large{Aprendizagem Automática}
        \vskip 3cm
        

 
        \Huge{\textbf{Lab 4}}
        \vskip 0.5cm

        \huge{\textbf{Bayes Classifier}}
        \vskip 0.5cm

 
        \vfill
 
        \large
        \textbf{Grupo 9}\\
        \vspace{0.3cm}
        Manuel Diniz, 84125\\
        Alexandre Rodrigues, 90002\\
        
        \vspace{1cm}

        \textbf{Turno:} 4ªf 11h00

    \end{center}
\end{titlepage}

\tableofcontents
\newpage

\chapter{Classificador de Bayes}

\chapter{Exemplo}

\chapter{Reconhecimento de Linguagem}

    \par É agora aplicado um \textit{naive Bayes classifier} a texto, de modo a se fazer o reconhecimento da linguagem em que este está escrito. O conjunto de treino é um conjunto de trigramas e o respetivo número de ocorrências nos textos originais.

    \section{Geração da matriz de treino}
        \par A partir dos dados, é gerada uma matriz de treino, com uma amostra (neste caso uma linguagem) em cada linha, e uma \textit{feature} (neste caso, o número de ocorrências de um determinado trigrama) por coluna. Gera-se também um vetor com as \textit{labels} corretas correspondentes às amostras.

    \section{Treino e teste}
        \par De seguida, o modelo é treinado e testado com os mesmos dados de treino, verificando que atribui a linguagem correta a cada conjunto de treino.
        \par Testa-se agora o modelo em 6 frases fornecidas. Estas são processadas de modo a serem decompostas nos seus trigramas, e gera-se um vetor linha com o mesmo formato dos dados de treino, ou seja, com o número de ocorrências de cada trigrama em cada coluna (fazendo correspondência posicional entre os trigramas obtidos e os do conjunto de treno). Os resultados da previsão neste conjunto de dados estão na tabela

        \begin{table}[H]
            \centering
            \begin{tabular}{|
            >{\columncolor[HTML]{C0C0C0}}l |c|c|c|c|}
            \hline
            \multicolumn{1}{|c|}{\cellcolor[HTML]{C0C0C0}Texto} & \cellcolor[HTML]{C0C0C0}\begin{tabular}[c]{@{}c@{}}Linguagem\\ real\end{tabular} & \cellcolor[HTML]{C0C0C0}\begin{tabular}[c]{@{}c@{}}Linguagem\\ reconhecida\end{tabular} & \cellcolor[HTML]{C0C0C0}\textit{Score} & \cellcolor[HTML]{C0C0C0}\begin{tabular}[c]{@{}c@{}}Margem de\\ classificação\end{tabular} \\ \hline
            Que fácil es comer peras. & es & es & 0.6703 & 0.3407 \\ \hline
            Que fácil é comer peras. & pt & pt & 0.9999 & 0.9999 \\ \hline
            Today is a great day for sightseeing. & en & en & 1.0000 & 1.0000 \\ \hline
            Je vais au cinéma demain soir. & fr & fr & 0.9999 & 0.9999 \\ \hline
            Ana es inteligente y simpática. & es & es & 0.9999 & 0.9999 \\ \hline
            Tu vais à escola hoje & pt & fr & 0.7930 & 0.5861 \\ \hline
            \end{tabular}
        \end{table}

        \par Todas as frases são identificadas corretamente com a exceção da última. A primeira é identificada com um \textit{score} modesto, pois não existem traços fortes que a língua em questão seja espanhol e não português. A única diferença que a frase tem da seguinte é a palavra "es", forma os trigramas " es" e "es ", que também são bastante comuns na língua portuguesa ("estar" ou "antes", por exemplo).
        \par O mesmo não é o caso na segunda frase, que tem um forte indicador da língua portuguesa, o trigrama " é ", que ocorre muito menos vezes no conjunto de treino espanhol. Deste modo, classifica a frase como português com uma confiança, ou \textit{score}, muito mais elevada.
        \par Mais uma vez a terceira frase é classificada com elevada confiança, devido à presença de trigramas que são fortes indicadores da sua língua, como "ght" ou "day" neste caso.
        \par A quarta frase também possui \textit{score} elevado, com um dos trigramas chave sendo "oir", muito comum no francês.
        \par A quinta possui um trigrama quase do exclusivo do espanhol, " y ", sendo que também é classificada com elevada segurança.
        \par A sexta e última frase é identificada incorretamente como francês, se bem que com um \textit{score} e margem de classificação reduzidos. Pode-se atribuir parcialmente à presença do trigrama " à ", que é muito comum no francês, para além de que quase todos os restantes trigramas são partilhados pelas duas linguagens de forma comum.

\end{document}
